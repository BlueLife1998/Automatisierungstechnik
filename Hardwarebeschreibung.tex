% !TEX root = EntwicklerdokumentationM5StickAlarm.tex

\section{Beschreibung der Hardware}
\subsection{Materialliste}

\vspace{1cm}
%\centering

\begin{table}[H]
	\centering
	
	\resizebox{420pt}{!}{
	\begin{tabular}{||p{0,8cm}|p{2,5cm}|p{2,7cm}|l|p{2,0cm}|r|}
		\hline
		Anz. & Bezeichnung               & Artikel-Nr.       	& Einzelpreis   & Bem.					& Bestellink 						  \\ 
		\hline
		\hline
		1      & Raspberry Pi 3B+          & RASPBERRYPI 3 B+ 	& 46,20 € &   Alternativ wäre auch Raspberry Pi 4B in Ordnung.                       		&\href{\linkPi}{reichelt.de}    	                             \\
		\hline
		1      & Raspberry Pi - Netzteil   & RASP NT 25 SW E      & 7,25 €  &   Falls auf Raspberry Pi 4B gewechselt anderes Netzteil verwenden. Bestenfalls originales Pi Netzteil mit USB Typ-C \href{\linkNTPiVier}{(Link: reichelt.de)}      	               		&\href{\linkPiNT}{reichelt.de}  	                             \\
		\hline
		1      & M5StickC                  &                   	& 29,95€        & Nur M5 Stick C+ verfügbar        &\href{\linkMstick}{eckstein-shop.de} 	                             \\
		\hline
		1      & Ethernet Kabel            & SLIM UL6 1,5 GR 	& 4,10€        & Auch anderes Kabel in Ordnung, wenn bereits bestellt und lagernd            					&\href{\linkLankabel}{reichelt.de}           		 				                           \\
		\hline
		1      & Piezo Buzzer              & DEBO BUZZER A2     	& 1,45 €  		&             					&\href{\linkBuzzer }{reichelt.de}                                \\
		\hline
		1      & Micro SD Karte 16Gb       &  TS16GUSDCU1       &7,95€         &      					& \href{\linkmicrosd}{reichelt.de} \\ 
		\hline
			1      & Taster              &DEBO BUTTON2         &2,20€         &     					&\href{\linkTaster}{reichelt.de}  \\ 
		\hline
		1      & Jumperkabel set              &DEBO KABELSET12        &1,70€         &     					&\href{\linkJumpercable}{reichelt.de}  \\ 
		\hline
		          						                  
	\end{tabular}
}

\caption{Materialliste für das Projekt Überwachungssystem basierend auf der IMU des M5StickC}	
\end{table}

Anmerkung: Die Versandkosten betragen bei Reichelt 5,95€ und bei Eckstein abhängig von der Versandart 1,99€ (nicht verfolgbar, nicht versichert) oder 3,50€ (verfolgbar, versichert).\par Die Versanddauer, die Verfügbarkeit sowie der Kaufpreis können sich aufgrund aktueller Begebenheiten innerhalb kurzer Zeit ändern. Es fällt kein Mindermengenzuschlag an.\par  
Alle Internetlinks und Preise wurden am 20.04.2022 zuletzt abgerufen. Der aktuelle Gesamtpreis der Bestellung liegt bei 100,80€ zzgl. Versandkosten.


\newpage

\subsection{M5StickC}
\subsubsection{Aufbau des M5StickC}
Der M5StickC ist in einem Kunststoffgehäuse (PC - Polycarbonate) untergebracht. Der Stick verfügt über zwei Taster, die je nach Anwendung frei programmiert werden können. Der Taster A, der in der Abbildung \ref{AufbauM5StickC} zu sehen ist, befindet sich auf der rechten langen Seite und der Taster B auf der Frontseite. Des Weiteren hat der M5Stick ein farbiges Display mit einer Auflösung von 80x160 Pixel bzw. 0,96“ Bildschirmdiagonale. Auf der linken langen Seite befindet sich die AN / AUS / Restart-Taste. Hierüber wird der M5StickC bei zweisekündiger Betätigung ein- und bei sechssekündiger Betätigung ausgeschaltet.  An den beiden kurzen Seiten sind die Schnittstellen (USB-Typ-C, Grove) sowie eine im Raster von 2,54 mm, 8 polige Buchsenleiste für Jumperkabel verbaut. Grove ist ein benutzerfreundliches System, mit dem sich ein Prozessor (z. B. Arduino) mit einer Vielzahl von Modulen, Sensoren und Eingängen verbinden lässt. Auf der Rückseite wird die Pinbelegung der Buchsenleiste dargestellt. \footnote{\cite{M5STACK.}}

\begin{figure}[H]	%[H] um an dieser Stelle anzuzeigen
\begin{center}
\includegraphics[width=10cm]{AufbauM5StickC.jpg}
\caption{Aufbau des Entwicklungsboardes „M5StickC" \protect\cite{M5STACK.}}
\label{AufbauM5StickC}
\end{center}
\end{figure}

\newpage



\subsubsection{Technische Daten des M5StickC}
\begin{itemize}
\item Eingebauter Mikrocontroller ESP 32 mit 240 MHz Dual-Core, 600 DMIPS, 520 KB SRAM
\item Flash Speicher von 4 MB
\item Akku – 95 mAh mit 3,7 V
\item Einsatztemperatur von 0...60 °C
\item Infrarot Übertragung
\item Abmaße (L x B x H) – 48,2 mm x 25,5 mm x 13,7 mm
\item 5 V Ein- und Ausgangsspannung
\end{itemize}

%IMUAchsen
\subsubsection{Verbaute Sensoren}
In dem M5StickC ist ein digitales Mikrofon (SPM1423) und eine Echtzeituhr (BM8563), auch RTC
(Real-Time-Clock) genannt, verbaut. Des Weiteren hat der Stick eine 6-Achsen-IMU (MPU-6886)
verbaut \footnote{\cite{M5STACK.}}. Die Abkürzung IMU steht für „Inertial Measurement Unit“, zu Deutsch
„Inertiale Messeinheit“. Die IMU ist eine Messeinheit, die aus der Kombination aus einem 3-Achsen-Beschleunigungssensors und einem 3-Achsen-Gyroskopsensors besteht. Die Beschleunigung
wird in der X, Y und Z-Richtung gemessen (s. Abb. \ref{IMUAchsen}). Für jede Achse wird eine separate Prüfmasse
verwendet, die bei Beschleunigung oder Verzögerung einen Weg zurücklegt. Der Abstand, der sich aus
dem Weg nach der Beschleunigung minus dem Weg vor der Beschleunigung ergibt, bestimmt die
Kapazität. Die Kapazität wird anschließend von einem Aufnehmer erkannt, in einen Messwert
umgerechnet und als Spannungssignal ausgegeben. Der Endwert des Ausgangs kann auf ±2g, ±4g,
±8g oder ±16g programmiert werden. \footnote{\cite{InvenSense.}}

Bei dem Gyroskop wird zunächst die Drehung um die dazugehörige Achse (s. Abb. \ref{IMUAchsen}) gemessen.
Wenn die Kreisel des Gyroskops um eine Achse gedreht werden, verursacht der Coriolis-Effekt eine
Schwingung, die anschließend von einem kapazitiven Aufnehmer erkannt wird. Das aufgenommene
Signal wird darauf verstärkt und gefiltert, um eine Spannung zu erzeugen. Diese Spannung ist
proportional zu der Winkelgeschwindigkeit des Gyroskops. Die Winkelgeschwindigkeit lässt sich
digital auf ±250, ± 500, ± 1000 oder ±2000 °
s
(dps=degrees per second) programmieren. \footnote{\cite{InvenSense.}}

\begin{figure}[H]	%[H] um an dieser Stelle anzuzeigen
\begin{center}
\includegraphics[width=10cm]{IMUAchsen.jpg}
\caption{Beschleunigungs- und Drehachsen der IMU
\protect\cite{InvenSense.}}
\label{IMUAchsen}
\end{center}
\end{figure}

\newpage

\subsection{Raspberry Pi 4}
Der Raspberry Pi ist ein opensource Einplatinencomputer, welcher ursprünglich als kostengünstiges System für Lernzwecke im Bereich der Programmierung gedacht war. \footnote{\cite{RaspberryPi.20.08.2015}} \par
Besonders in der Makerszene erfreut sich der Raspberry Pi (Abb. \ref{pi_mitgehaeuse}) einer großen Beliebtheit. Es sind Projekte vom Desktopcomputer \footnote{\cite{RaspberryPi.20.08.2015}} bis hin zur Steuerung und Überwachung heimischer 3D-Drucker \footnote{\cite{Hauge.20.04.2022}} realisierbar.\\[3mm]

	\begin{figure}[H]
	\centering
	\includegraphics[height=250pt]{Pi_4.jpg}
	\caption{Raspberry Pi 4B}
	\label{pi_mitgehaeuse}
\end{figure}	

\subsubsection{Raspberry Pi für das Projekt}
Für das Projekt Überwachungssystem mittels der IMU des M5 Sticks soll ein 
Raspberry Pi 4B in der Version mit 1GB RAM als Server fungieren auf welchem der Mosquitto MQTT Broker operieren soll. Des Weiteren soll der Server selbst als Aktor für den Alarm dienen.\par

Der Raspberry Pi 4B wurde für das Projekt als am besten geeignet eingestuft. Er verfügt über einen Gigabit-LAN Anschluss welcher mit Hilfe eines Ethernetkabels für eine stabile Verbindung mit dem Router des Heimnetzwerkes sorgen soll. Verglichen mit älteren Raspberry Pi Modellen ist der Pi 4B mit einer etwas stärkeren CPU, sowie einer WLAN und Bluetooth Karte ausgestattet, welche die Möglichkeiten bieten das Projekt um mehr IoT Geräte, beispielsweise Sensoren und Aktoren, zu erweitern. Zudem ist der preisliche Unterschied (UVP) zu den Vorgängermodellen marginal.  

	// \begin{figure}[H]
	// \centering
	//\includegraphics[height=200pt]{tabelle_pi_vergleich}
	//\caption{Vergleich der verschiedenen Raspberry Pi Modelle \protect\cite[S. 59]{Huwe.2019}}
	// \label{tabelle_pi_vergleich}
// \end{figure}

\subsubsection{Anschlussmöglichkeiten am Raspberry Pi 3B+}
Der Raspberry Pi 4B verfügt im Allgemeinen über folgende Anschlussmöglichkeiten (siehe Abb. \ref{RaspiAnschluesse}):\footnote{\cite[S. 60]{Huwe.2019}}\par  
//NEUE QUELLE: HANDBUCH PI4!!!!!!!!!!!  //https://datasheets.raspberrypi.com/rpi4/raspberry-pi-4-product-brief.pdf
$\bullet$ 2x micro-HDMI   \par %\hspace{20pt} $\bullet$ 
$\bullet$ 3,5mm AUX\par 
$\bullet$ 4 $\times$ USB 2.0 \par 
$\bullet$ Ethernet 1Gbit/RJ45\par 
$\bullet$ CCI (Camera Connector Interface)\par 
$\bullet$ DSI (Display Serial Interface)\par
$\bullet$ 40 GPIO-Pins\par
$\bullet$ Steckplatz für die microSD-Karte\par
$\bullet$ USB-C Anschluss für die Stromversorgung des PIs\par

	\begin{figure}[H]
		\centering
		\includegraphics[width=15cm]{RaspiAnschluesse}
		\caption{Hardwarebezeichnung auf Pi mit Beschriftung \cite{notebooksbilliger.deAG.21.04.2022}}
		\label{RaspiAnschluesse}
	\end{figure}
	
Für das Projekt sind insbesondere die GPIO-Pins und der Ethernetport von Bedeutung.\par

\newpage

\subsection{Weitere Hardware}
\subsubsection{Piezo Buzzer Modul}
	Um beim Auslösen der Alarmanlage ein Warnsignal zu erzeugen, soll ein aktives Piezo Buzzer Modul ST1143 der Marke Iduino (Abb. \ref{ST1143}) eingesetzt werden.\par Die Pins des Buzzers werden wie folgt mittels Jumperkabeln mit den GPIO Pins des Pis verbunden (Tab. \ref{BuzzerTabelle}, Abb. \ref{BuzzerAnschluss}): \footnote{\cite{Draeger.2019}}
	
	\begin{table}[H]
		\centering
		\begin{tabular}{|p{5cm}|p{5cm}|} 
			\hline
			Buzzer & Raspberry Pi\\ 
			\hline
			SIG (S)   & GPIO 4 \hspace{0,1cm}(Pin 7)\\  
			\hline
			VCC~ ($+$; mittlerer Pin)   & 3,3V 	\hspace{0,7cm}(Pin 1)\\
			\hline
			GND (-)    & Ground  	\hspace{0,2cm}(Pin 6)\\
			\hline
		\end{tabular}
	\caption{Verbindung ST1143 mit Pi}
	\label{BuzzerTabelle}
	\end{table} 

	\begin{figure}[H]
	\centering
	\includegraphics[width=10cm]{BuzzerAnschluss}
	\caption{Schaltplan ST1143 an Raspberry Pi}
	\label{BuzzerAnschluss}
	\end{figure}

	\begin{figure}[H]	%H
	\centering
	\includegraphics[height=150pt]{ST1143}
	\caption{Aktiver Piezo Buzzer ST1143 \cite{reicheltelektronikGmbH&amp.21.04.2022}}
	\label{ST1143}
	\end{figure}

\subsubsection{Taster}

Der Alarm soll über ein Tastermodul SBC-BUTTON2 (Abb. \ref{BUTTON2}) mit jeweils einem Button deaktiviert beziehungsweise manuell aktiviert werden können.
Das Modul wird mittels Jumperkabeln mit den GPIO Pins des Pis verbunden (siehe Schaltplan Abb. \ref{TasterAnschluss} und Tabelle \ref{TasterTabelle}). 
In der gewählten Anschlusskonfiguration wird das Modul als High-Aktiv betrieben. Dies bedeutet, dass bei Tastendruck der GPIO23 bzw. GPIO24 mit der Spannungsquelle verbunden wird und der Raspberry Pi jeweils ein HIGH Signal erhält. \footnote{\cite{.19.04.2022}}\par 
Es wäre ebenfalls möglich durch vertauschen des VCC und GND Anschlusses das Modul als Low-Aktiv zu betreiben, hierbei würde der Pi jeweis ein LOW Signal erhalten.

\begin{table}[H]
	\centering
	\begin{tabular}{|p{5cm}|p{5cm}|} 
		\hline
		Tastermodul & Raspberry Pi\\ 
		\hline
		SIG(S1)                   & GPIO 23 \hspace{0,3cm}(Pin 16)\\
		\hline
		SIG(S2)                   & GPIO 24 \hspace{0,3cm}(Pin 18)\\
		\hline
		VCC~(+)   & 3,3 V \hspace{1cm}(Pin\hspace{0,35cm}1)\\ 
		\hline
		GND(-)                   & Ground \hspace{0,6cm}(Pin\hspace{0,35cm}6)\\
		\hline
		
\end{tabular}
	\caption{Verbindung Taster SE043 mit Pi}
	\label{TasterTabelle}
\end{table}

	\begin{figure}[H]
	\centering
	\includegraphics[width=15cm]{TasterAnschluss}
	\caption{Schaltplan Taster SE043 an Raspberry Pi}
	\label{TasterAnschluss}
	\end{figure}

	\begin{figure}[h] %H
	\centering
	\includegraphics[height=150pt]{BUTTON2}
	\caption{Joy-it SBC-BUTTON2 \cite{reicheltelektronikGmbH&amp.21.04.2022b}}
	\label{BUTTON2}
	\end{figure}

\subsubsection{Zubehör des Raspberry Pi}
Für die Spannungsversorgung soll ein originales Raspberry Pi USB-C-Netzteil mit einer maximal abrufbaren Leistung von 13W genutzt werden.\par

Eine MicroSDHC-Speicherkarte mit einer Kapazität von 16GB und einer Lesegeschwindigkeit von bis zu 90 MB/s soll den nötigen Speicher für den Mosquitto MQTT Broker und weitere Aufgaben bereitstellen. \href{\linkmicrosd}{(Reichelt Bestelllink)} 

\newpage

\subsection{Anschlussplan Raspberry Pi}
Der Anschluss der Hardwarebauteile an den Pi erfolgt über die auf dem Pi verbauten Pins laut Anschlussplan (Abb. \ref{Anschlussplan}). Der Raspberry Pi benötigt einen Anschluss per USB-C zur Spannungsversorgung. Dies kann, je nach Anliegen, entweder über den Router erfolgen, sofern dieser über eine USB-Schnittstelle verfügt, oder über einen externen Spannungswandler (vgl. Handyladegerät). Der Warnsummer/Buzzer benötigt keine externe Spannungsquelle und wird direkt über den Pi mit Spannung versorgt.

\begin{figure}[H]	%[H] um an dieser Stelle anzuzeigen
\begin{center}
\includegraphics[width=15cm]{Anschlussplan.png}
\caption{Anschlussschematik Raspberry Pi 3}
\label{Anschlussplan}
\end{center}
\end{figure}
